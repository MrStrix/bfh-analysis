%Simon Krenger: Analysis
\documentclass{report}
\usepackage{amsmath}
\usepackage{amsthm}
\usepackage{amssymb}
\usepackage[utf8]{inputenc} 
\usepackage{graphicx}
\usepackage{mathtools}

\usepackage{color}
\definecolor{red}{rgb}{1,0,0}

\newtheorem{mydef}{Definition}
\newtheorem{myexample}{Beispiel}
\newtheorem{myproof}{Beweis}
\newtheorem{axiom}{Axiom}

\title{Analysis}
\author{Simon Krenger}

\begin{document}
\maketitle
\chapter{Folgen und Grenzwerte}
\section{Folgengesetze}
Betrachten wir Zahlenfolgen wie
\begin{enumerate}
\item $5, 9, 13, 17, 21, ...$ \label{folgen:enum1}
\item $9, 3, 1, \frac{1}{3}, \frac{1}{9}, \frac{1}{27}, ...$ \label{folgen:enum2}
\item $\frac{1}{8}, -\frac{1}{4}, \frac{1}{2}, -1, 2, -4, ... $ \label{folgen:enum3}
\item $1, 1, 2, 3, 5, 8, 13, 21, ...$ \label{folgen:enum4}
\end{enumerate}
So fällt einerseits die Gesetzmässigkeit und andererseits ein immer vorhandenes erstes Element auf.
\begin{mydef}Eine \underline{Folge} ist eine Funktion mit $\mathbb{N}$ als Definitionsmenge\end{mydef}
Für $f: \mathbb{N} \mapsto \mathbb{R}$ mit $y = f(x)$ schreiben wir bei Folgen
\begin{equation}(a_n)_{n \in \mathbb{N}} \quad \mbox{mit} \quad a_n=a(n)\end{equation}
\begin{myexample}$a_n = 3n + 7$ ist das Gesetz für die Folge $10, 13, 16, ...$\end{myexample}
Das Gesetz einer Folge können wir so angeben, dass auf das vorangehende \underline{Folgenglied} (Element) Bezug genommen wird. So erhalten wir das \underline{rekursive Gesetz}.
\\\\
Bei (\ref{folgen:enum1}) lautet dies $a_{n+1} = a_n + 4$ \underline{und} $a_1 = 5$ und bei (\ref{folgen:enum2}) $b_{n+1}=\frac{1}{3}b_n$ und $b_1=9$.
\\\\
Wir können aber auch ein Gesetz suchen, welches $a_n$ mit Hilfe von $n$ berechnet. Das nennen wir das \underline{explizite Gesetz}.
\\\\
Die Folge (\ref{folgen:enum3}) ist eine \underline{alternierende} Folge (abwechselnd $+$ und $-$). Dann muss $(-1)^n$ oder $(-1)^{n+1}$ im expliziten Gesetz stehen.\\
Bei (\ref{folgen:enum3}) also $a_n = (-1)^{n+2} \cdot 2^{n-4}$\\
Für die Folge  (\ref{folgen:enum4}) finden wir das rekursive Gesetz
\begin{equation}a_{n+1} = a_n + a_{n-1} \quad \mbox{und} \quad a_1=1, a_2=1\end{equation}
für die \underline{Fibonacci-Folge}. Das explizite Gesetz ist schwierig zu finden.

\section{Teilsummen}
Wenn wir die Elemente einer Folge addieren, so erhalten wir die n-te Teilsumme
\begin{equation}s_n = a_1 + a_2 + a_3 + ... + a_n \quad (n \in \mathbb{N})\end{equation}
\begin{equation}s_n = \sum_{k=1}^{n} a_k \quad \mbox{("Summe $a_k$, k von 1 bis n")}\end{equation}
Wollen wir zum Beispiel
\begin{equation}1 + 2 + 3 + 4 + ... + 100\end{equation}
berechnen, so greifen wir auf die Idee von Carl Friedrich Gauss (1777 bis 1855, Göttingen) zurück.
\begin{eqnarray}s_{100} & = & (1 + 100) + (2 + 99) + ... + (50 + 51) \nonumber \\
& = & 50 \cdot 101 \nonumber \\
& = & 5050\end{eqnarray}
Wollen wir
\begin{equation}s_n = \frac{1}{2 \cdot 5} + \frac{1}{5 \cdot 8} + \frac{1}{8 \cdot 11} + ... + a_n\end{equation}
berechnen, so bestimmen wir die \underline{Teilsummenfolge}.
\begin{eqnarray}s_1 & = & \frac{1}{10}\\
s_2 & = & \frac{1}{10} + \frac{1}{5 \cdot 8} = \frac{8 + 2}{2 \cdot 5 \cdot 8} = \frac{10}{80} = \frac{1}{8} \textcolor{red}{ = \frac{2}{16}}\\
s_3 & = & \frac{1}{8} + \frac{1}{8 \cdot 11} = \frac{11 + 1}{8 \cdot 11} = \frac{12}{8 \cdot 11} = \frac{3}{22}\\
s_4 & = & \frac{3}{22} + \frac{1}{11 \cdot 14} = \frac{42 + 2}{2 \cdot 11 \cdot 14} = \frac{2}{14} = \frac{1}{7} \textcolor{red}{ = \frac{4}{28}}\end{eqnarray}
also finden wir
\begin{equation}s_n = \frac{n}{6n + 4}\end{equation}
Wir beweisen mit vollständiger Induktion.
\begin{myproof}\begin{enumerate}
\item \label{proof:teilsummenfolge-1} Behauptung ist für $n=1$ wahr, denn
\begin{equation}s_1 = \frac{1}{10}\end{equation}
und $s_n$ wird mit $n=1$ zu
\begin{equation}s_1 = \frac{1}{6 \cdot 1 + 4} = \frac{1}{10}\end{equation}
\item \label{proof:teilsummenfolge-2} Voraussetzung: 
\begin{equation}s_n = \frac{n}{6n+4}\end{equation}
Behauptung:
\begin{equation}s_{n+1} = \frac{n+1}{6(n+1) + 4} = \frac{n+1}{6n+10}\end{equation}
Beweis:
\begin{equation}s_{n+1} = s_n + a_{n+1}\end{equation}
und wir müssen $a_{n+1}$ bestimmen. Es ist
\begin{equation}a_n = \frac{1}{(3n-1)(3n+2)}\end{equation}
, also
\begin{eqnarray}a_{n+1} & = & \frac{1}{(3(n+1)-1)(3(n+1)+2)} \\
& = & \frac{1}{(3n+2)(3n+5)}\end{eqnarray}
Somit ist
\begin{eqnarray}s_{n+1} & = & \frac{n}{6n+4} + \frac{1}{(3n+2)(3n+5)} \\
& = & \frac{n}{2(3n+2)} + \frac{1}{(3n+2)(3n+5)} \\
& = & \frac{3n^2+5n+2}{2(3n+2)(3n+5)} \\
& = & \frac{(3n+2)(n+1)}{2(3n+2)(3n+5)} \\
& = & \frac{n+1}{2(3n+5)} = \frac{n+1}{6n+10}\end{eqnarray}
\item Nach (\ref{proof:teilsummenfolge-1}) ist die Behauptung für $n=1$ wahr und nach (\ref{proof:teilsummenfolge-2}) für $n+1$, also für $n=2$ usw. Also ist die Behauptung für alle $\mathbb{N}$ wahr.
\end{enumerate}\end{myproof}
Natürlich finden wir auch sofort
\begin{equation}\sum_{k=1}^{n} k = 1 + 2 + ... + n = (1+n)\frac{n}{2}\end{equation}

\section{Grenzwerte}
Zeichnen wir die Elemente von
\begin{equation}a_n = (-1)^n(2+\frac{1}{n})\end{equation}
auf der Zahlengeraden,
\\\\TODO\\\\
so häufen sich die Werte bei 2 und -2. Sowohl in der Nähe von 2 als auch -2 liegen unendlich viele Werte.
\begin{mydef}Wir nennen
\begin{equation}U_\epsilon(a) := [a-\epsilon; a + \epsilon]; \epsilon \in \mathbb{R}^+\end{equation}
eine \underline{$\epsilon$-Umgebung} von $a$.\end{mydef}
\begin{mydef}Finden wir in jeder $\epsilon$-Umgebung einer Zahl $a$ unendlich-viele Folgenglieder, so heisst $a$ ein Häufungswert der Folge $(a_n)$.\end{mydef}
\begin{myexample}\begin{enumerate}
\item \begin{eqnarray}a_n & = & \frac{n+1}{n} \nonumber \\
& \rightarrow & 2, 1.5, 1.\bar{3}, 1.25, 1.2, 1.16, 1.\bar{142857}\end{eqnarray}
also ist 1 ein Häufungswert.
\item \begin{eqnarray}b_n & = & 4n + 1 \nonumber \\
& \rightarrow & 5, 9, 13, 17, ...\end{eqnarray}
Also kein Häufungswert.\end{enumerate}\end{myexample}
Besitzt eine Folge genau einen einzigen Häufungswert, so nennen wir diesen den \underline{Limes} (Grenzwert) der Folge und schreiben
\begin{equation}\lim_{n \to \infty}a_n = a \end{equation}
\begin{myexample}\begin{enumerate}
\item \begin{equation}\lim_{n \to \infty}\frac{n+1}{n}=1\end{equation}
\item \begin{equation}\lim_{n \to \infty}(-1)^n(2+\frac{1}{n})\end{equation}
\item \begin{equation}\lim_{n \to \infty}\frac{1}{n}=0\end{equation}
\end{enumerate}\end{myexample}
\begin{mydef}Im Folgenden die Definition des Limes nach Cauchy (Louis Augustin Cauchy, 1789 bis 1857, Paris, Schweiz, Prag):\\
Eine Zahlenfolge $(a_n)$ besitzt den Grenzwert $a$, wenn für alle $\epsilon > 0$ von einem bestimmten $n$ an, alle weiteren Folgenglieder in $U_\epsilon(a)$ liegen:
\begin{equation}\epsilon \in \mathbb{R}, n_0 \in \mathbb{N}: \forall \epsilon > 0\exists n_0 \forall n (n > n_0 \to |a_n-a| < \epsilon)\end{equation}\end{mydef}
\begin{myexample}$a_n = \frac{5}{n}$ hat Grenzwert $a=0$\\
Ist $\epsilon = \frac{1}{100}$, so wird $n_0 = 500$, denn von $a_{501}$ an sind alle Folgenglieder zwischen $0,01$ und $0$.\end{myexample}
\begin{mydef}Besitzt eine Folge einen Grenzwert, so heisst sie \underline{konvergent}, andernfalls \underline{divergent}. Folgen mit Grenzwert $0$ heissen \underline{Nullfolgen}.\end{mydef}
\begin{myexample}Wir untersuchen diese Folgen:\begin{enumerate}
\item \begin{equation}\lim_{n \to \infty}\frac{1}{n}=0, \lim_{n \to \infty}\frac{k}{n} = 0, k \in \mathbb{Z} \setminus \{0\}\end{equation}
\item \begin{equation}\lim_{n \to \infty}\frac{1}{n^2}=0, \lim_{n \to \infty}\frac{k}{n^2} = 0, k \in \mathbb{Z} \setminus \{0\}\end{equation}
\item \begin{equation}\lim_{n \to \infty}\frac{1}{n^k}=0, k \in \mathbb{N}\end{equation}
\item \begin{equation}\lim_{n \to \infty}2^n\end{equation} existiert nicht, die Folge $a_n = 2^n$ \underline{wächst über alle Schranken}; sie ist also divergent.
\item \begin{equation}\lim_{n \to \infty}(2+\frac{1}{n}) = \lim_{n \to \infty}2 + \lim_{n \to \infty}\frac{1}{n} = 2\end{equation}
\item \begin{equation}\lim_{n \to \infty}\frac{3n^2+5}{n^2} = \lim_{n \to \infty}3 + \lim_{n \to \infty}\frac{5}{n^2} = 3\end{equation}
\item \begin{equation}\lim_{n \to \infty}\frac{4n-7}{3n+1} \neq \frac{\lim_{n \to \infty}4n-7}{\lim_{n \to \infty}3n+1}\end{equation}
\end{enumerate}\end{myexample}
\newpage
Wir brauchen die \underline{Grenzwertsätze}. Sind $(a_n)$ und $(b_n)$ konvergent mit $\lim_{n \to \infty}a_n = a, \lim_{n \to \infty}b_n = b$, so ist
\begin{enumerate}
\item \begin{equation}\lim_{n \to \infty}(a_n + b_n) = \lim_{n \to \infty}a_n + \lim_{n \to \infty}b_n = a + b\end{equation}
\item \begin{equation}\lim_{n \to \infty}(a_n \cdot b_n) = \lim_{n \to \infty}a_n \cdot \lim_{n \to \infty}b_n = a \cdot b\end{equation}
\item \begin{equation}\forall n (b_n \neq 0) \land b_n \neq 0 :\\
\lim_{n \to \infty}\frac{a_n}{b_n} = \frac{\lim_{n \to \infty}a_n}{\lim_{n \to \infty}b_n} = \frac{a}{b}\end{equation}
\end{enumerate}
Um
\begin{equation}\lim_{n \to \infty}\frac{5n-2}{6n+1}\end{equation}
zu berechnen, überlegen wir, dass Brüche gekürzt werden können und kürzen mit n.
\begin{eqnarray}\lim_{n \to \infty}\frac{5n-2}{6n+1} & = & \lim_{n \to \infty}\frac{\frac{5n-2}{n}}{\frac{6n+1}{n}} = \lim_{n \to \infty}\frac{\frac{5n}{n}-\frac{2}{n}}{\frac{6n}{n}+\frac{1}{n}} \nonumber \\
= \lim_{n \to \infty}\frac{5-\frac{2}{n}}{6+\frac{1}{n}} & =  &\frac{\lim_{n \to \infty}(5-\frac{2}{n})}{\lim_{n \to \infty}(6+\frac{1}{n})} = \frac{5}{6}\end{eqnarray}
\begin{myexample}Wir kürzen jeweils mit der Variabel ($n$) mit der höchsten Potenz:\begin{enumerate}
\item \begin{eqnarray}\lim_{n \to \infty}\frac{2n^2+1}{4n^2-3} & = & \lim_{n \to \infty}\frac{\frac{2n^2}{n^2}+\frac{1}{n^2}}{\frac{4n^2}{n^2}-\frac{3}{n^2}} \nonumber \\ 
= \lim_{n \to \infty}\frac{2+\frac{1}{n^2}}{4-\frac{3}{n^2}} & = & \frac{\lim_{n \to \infty}(2+\frac{1}{n^2})}{\lim_{n \to \infty}(4-\frac{3}{n^2})} = \frac{2}{4} = \frac{1}{2}\end{eqnarray}
\item \begin{equation}\lim_{n \to \infty}\frac{n^2+2n+5}{n-7} = \lim_{n \to \infty}\frac{1+\frac{2}{n} + \frac{5}{n^2}}{\frac{1}{n} - \frac{7}{n^2}}\end{equation}
existiert nicht.
\item \begin{eqnarray}\lim_{n \to \infty}\frac{\sqrt{n}+\sqrt{n-1}}{\sqrt{n}-\sqrt{2n-1}} & = & \lim_{n \to \infty}\frac{\frac{\sqrt{n}}{\sqrt{n}}+\frac{\sqrt{n-1}}{\sqrt{n}}}{\frac{\sqrt{n}}{\sqrt{n}}-\frac{\sqrt{2n-1}}{\sqrt{n}}} = \lim_{n \to \infty}\frac{1+\frac{\sqrt{n-1}}{\sqrt{n}}}{1-\frac{\sqrt{2n-1}}{\sqrt{n}}} \nonumber \\
= \lim_{n \to \infty}\frac{1+\sqrt{1-\frac{1}{n}}}{1-\sqrt{2-\frac{1}{n}}} &=& \frac{1+\sqrt{1}}{1-\sqrt{2}} = \frac{2}{1-\sqrt{2}}\end{eqnarray}
\end{enumerate}\end{myexample}
\newpage
Um
\begin{equation}\lim_{n \to \infty} (1+\frac{1}{n})^n\end{equation}
zu berechnen, bestimmen wir
\\\\$a_{100} = 2,70481...$, $a_{1000} = 2,7169...$\\\\
und finden, dass die Folge beschränkt ist. Es ist
\begin{equation}\lim_{n \to \infty} (1 + \frac{1}{n})^1 = e\end{equation}
und $e = 2,71828...$ ist irrational und transzendent.
\begin{myexample}Wir betrachten\begin{enumerate}
\item \begin{equation}\lim_{n \to \infty} ( 1 + \frac{1}{5n})^{5n} = e\end{equation}
\item \begin{equation}\lim_{n \to \infty} ( 1 + \frac{1}{\frac{n}{3}})^{\frac{n}{3}} = e\end{equation}
\item \begin{equation}\lim_{n \to \infty} ( 1 + \frac{1}{n})^{4n} = \lim_{n \to \infty}[ ( 1 + \frac{1}{n})^{n}]^4 = e^4\end{equation}
\end{enumerate}\end{myexample}
Wollen wir
\begin{equation}\lim_{n \to \infty} ( 1 + \frac{k}{n})^n, k \in \mathbb{Z}\end{equation}
berechnen, so überlegen wir, dass
\begin{equation}1 + \frac{k}{n} ) 1 + \frac{1}{\frac{n}{k}}\end{equation}
und finden
\begin{equation}\lim_{n \to \infty} ( 1 + \frac{1}{\frac{n}{k}})^\frac{n}{k} = e\end{equation}
Weiter ist dann
\begin{equation}\lim_{n \to \infty} (1 + \frac{k}{n})^n = \lim_{n \to \infty}[(1 + \frac{k}{\frac{n}{k}})^\frac{n}{k}]^k = e^k\end{equation}
und damit
\begin{equation}\lim_{n \to \infty} (1 - \frac{1}{n})^n = \lim_{n \to \infty} (1 + \frac{-1}{n})^n = e^{-1}\end{equation}
Zusammengefasst
\begin{eqnarray}\lim_{n \to \infty} (1 + \frac{1}{n})^n & = & e\\
\lim_{n \to \infty} (1 - \frac{1}{n})^n & = & \frac{1}{e} = e^{-1}\\
\lim_{n \to \infty} (1 + \frac{k}{n})^n & = & e^k, k \in \mathbb{Z}\end{eqnarray}

\section{Reihen}
\begin{mydef}Ist $(a_n)$ eine Folge, so heisst
\begin{equation}\sum_{n=1}^{\infty}a_n = a_1 + a_2 + a_3 + ...\end{equation}
eine \underline{Reihe}.\end{mydef}
Wollen wir
\begin{equation}\sum_{n=1}^{\infty}\frac{1}{n (n+1)} = \frac{1}{1 \cdot 2} + \frac{1}{2 \cdot 3} + \frac{1}{3 \cdot 4} + ...\end{equation}
bestimmen, so nützen wir
\begin{equation}\frac{1}{n(n+1)} = \frac{1}{n}-\frac{1}{n+1}\end{equation}
aus und erhalten so
\begin{eqnarray}\sum_{n=1}^{\infty} \frac{1}{n (n+1)} & = & \sum_{n=1}^{\infty} \frac{1}{n}-\frac{1}{n+1}\nonumber \\
& = & 1 - \frac{1}{2} + \frac{1}{2} - \frac{1}{3} + \frac{1}{3} - \frac{1}{4} + ... = 1\end{eqnarray}
Um die Reihe
\begin{equation}\sum_{n=1}^{\infty} \frac{1}{(3n-1)(3n+2)}\end{equation}
zu berechnen, bestimmen wir die \underline{Teilsummenfolge (Partialsummenfolge)}
\\\\$s_1, s_2, s_3, ...$\\\\
und überlegen dann, ob diese Folge einen Grenzwert besitzt.
\begin{eqnarray}s_n & = & \sum_{k=1}^{\infty} \frac{1}{(3k-1)(3k+2)} \nonumber \\
&=& \frac{1}{2 \cdot 5} + \frac{1}{5 \cdot 8} + \frac{1}{8 \cdot 11} + ... + a_n\end{eqnarray}
und so
\begin{eqnarray}s_1 & = & \frac{1}{10} \nonumber \\
s_2 & = & \frac{1}{10} + \frac{1}{5 \cdot 8} = \frac{8+2}{2 \cdot 5 \cdot 8} = \frac{10}{80} = \frac{1}{8} (= \frac{2}{16}) \nonumber \\
s_3 & = & \frac{1}{8} + \frac{1}{8 \cdot 11} = \frac{12}{8 \cdot 11} = \frac{3}{22}\end{eqnarray}
, also
\begin{equation}s_n = \frac{n}{6n+4}\end{equation}
und so
\begin{equation}\sum_{n=1}^{\infty} \frac{1}{(3n-1)(3n+2)} = \lim_{n \to \infty} \frac{n}{6n+4} = \frac{1}{6}\end{equation}
Wir sagen dann, dass die Reihe konvergent ist und der \underline{Wert der Reihe} ist $\frac{1}{6}$.
\begin{mydef}Wir nennen
\begin{equation}\sum_{n=1}^{\infty} a_1 q^{n-1} = a_1 + a_1 q + a_1 q^2 + ... \quad \mbox{mit} \quad q \neq 0,1\end{equation}
eine \underline{geometrische Reihe}.\end{mydef}
\begin{myexample}\begin{enumerate}
\item \begin{equation}\sum_{n=1}^{\infty} 2^{n-3} = \frac{1}{4} + \frac{1}{2} + 1 + 2 + 4 + ...\end{equation}
\item \begin{equation}\sum_{n=1}^{\infty} 4(\frac{1}{3})^{n-1} = 4 + \frac{4}{3} + \frac{4}{9} + \frac{4}{27} + ...\end{equation}
\end{enumerate}\end{myexample}
Es ist also der \underline{Quotient} $q \neq 0,1$ zweier aufeinanderfolgenden Glieder stets gleich. Um Den Wert der Reihe zu bestimmen, betrachten wir
\begin{eqnarray}s_n & = & a_1 + a_1 \cdot q + a_1 \cdot q^2 + ... + a_1 \cdot q^{n-1} \nonumber \\
- ( q \cdot s_n & = & a_1 \cdot q + a_1 \cdot q^2 + a_1 \cdot q^3 + ... + a_1 \cdot q^n ) \nonumber \\\\
s_n - q \cdot s_n & = & a_1 - a_1 \cdot q^n \nonumber \\
s_n (1-q) & = &  a_1(1-q^n)\end{eqnarray}
Damit
\begin{equation}s_n = \frac{a_1(1-q^n)}{(1-q)} \quad \mbox{mit} \quad q \neq 0,1\end{equation}
Dann ist
\begin{equation}\sum_{n=1}^{\infty} a_1 q^{n-1} = \lim_{n \to \infty} \frac{a_1(1-q^n)}{(1-q)}\end{equation}
und für $q > 1 \lor q < -1$ existiert der Grenzwert nicht. Für $-1 < q < 1$ ist
\begin{equation}\lim_{n \to \infty}\frac{a_1(1-q^n)}{1-q} = \frac{a_1}{1-q} \quad \mbox{, da} \quad \lim_{n \to \infty}q^n = 0\end{equation}
Also ist der Wert der geometrischen Reihe
\begin{equation}\sum_{n=1}^{\infty}a_1q^{n-1} = \frac{a_1}{1-q} \quad \mbox{mit} \quad -1 < q < 1\end{equation}
\begin{myexample}\begin{equation}2 - 1 + \frac{1}{2} - \frac{1}{4} + \frac{1}{8} - ... = \frac{2}{1-(-\frac{1}{2})} = \frac{4}{3}\end{equation}\end{myexample}
\begin{mydef}Wir nennen
\begin{equation}s = 1 + \frac{1}{2} + \frac{1}{3} + \frac{1}{4} + ...\end{equation}
die \underline{harmonische Reihe}.\end{mydef}
$s$ ist nicht konvergent, denn es ist
\begin{equation}1 + \frac{1}{2} = \frac{3}{2}\end{equation}
und
\begin{equation}\frac{1}{3} + \frac{1}{4} = \frac{7}{12} \textcolor{red}{ = 0,5837... > \frac{1}{2}}\end{equation}
und
\begin{equation}\frac{1}{5} + \frac{1}{6} + \frac{1}{7} + \frac{1}{8} = \frac{533}{840} \textcolor{red}{ = 0.634... > \frac{1}{2}}\end{equation}
also
\begin{equation}s = 1 + \frac{1}{2} + \frac{1}{3} +\frac{1}{4} +\frac{1}{5} +\frac{1}{6} +\frac{1}{7} +\frac{1}{8} +\frac{1}{9} + ...\end{equation}
So ist $s$ divergent! Hingegen ist
\begin{equation}1 + \frac{1}{2!} +\frac{1}{3!} +\frac{1}{4!} + ... = \sum_{k=0}^{\infty} \frac{1}{k!} = e\end{equation}
und
\begin{equation}1 + \frac{1}{2^2} +\frac{1}{3^2} +\frac{1}{4^2} + ... = \frac{\pi^2}{6}\end{equation}
und die \underline{Leibnizsche Reihe}
\begin{equation}1 - \frac{1}{3} +\frac{1}{5} -\frac{1}{7} + \frac{1}{9} - ... = \frac{\pi}{4}\end{equation}

\section{Stetigkeit}
Wir wollen für eine Funktion
\begin{equation}f: D_f \mapsto \mathbb{R}\end{equation}
den Grenzwert
\begin{equation}\lim_{x \to a} f(x)\end{equation}
bestimmen. Dazu brauchen wir eine Folge
\begin{equation}x_1, x_2, x_3, ...\end{equation}
mit Grenzwert $a$.
\\\\TODO: Graph\\\\
Nun bestimmen wir die Folge
\begin{equation}f(x_1), f(x_2), f(x_3), ...\end{equation}
und suchen deren Limes.
\\\\TODO: Graph\\\\
Bei der Funktion $g$ mit einer \underline{Sprungstelle bei $a$} hat
\begin{equation}g(x_1), g(x_2), ...\end{equation}
den Limes $b$. Aber
\begin{equation}g({x_1}'), g({x_2}'), ...\end{equation}
hat den Limes $c$. Also existiert
\begin{equation}\lim_{x \to a} g(x)\end{equation}
nicht. Beide Reihen $(x_1, x_2, x_3, ...)$ $({x_1}', {x_2}', {x_3}')$ müssen den gleichen Grenzwert besitzen.\\
Um
\begin{equation}\lim_{x \to a} f(x)\end{equation}
zu bestimmen, müssen \underline{alle} Folgen von x-Werten mit Grenzwert $a$ betrachtet werden. Wenn die Folgen der zugehörigen Funktionswerte alle denselben Limes besitzen, so ist dies $\lim_{x \to a} f(x)$.
\begin{myexample}\begin{enumerate}
\item \begin{equation}\lim_{x \to 1} x^2-2\end{equation}
TODO
\begin{equation}\lim_{x \to 1} x^2-2 = 1^2 -2 = -1\end{equation}
\item \begin{equation}\lim_{x \to 3} \frac{1}{x-3}\end{equation}
TODO\\\\
$g(x) = \frac{1}{x-3}$ besitzt einen \underline{Pol bei $x = 3$}, also existiert $\lim_{x \to 3} \frac{1}{x-3}$ nicht.
\item \begin{equation}\lim_{x \to 0} sgn(x) \quad \mbox{("Signum von x")}\end{equation}Es ist
\[
 sgn(x) := \begin{dcases*}
        -1  &, falls $x < 0$\\
        0 &, falls $x = 0$\\
	1 & , falls $x > 0$
        \end{dcases*}
\]
TODO\\\\
$f: \mathbb{R} \mapsto \{ -1, 0, 1\}$ mit $f(x) = sgn(x)$ besitzt für $x=0$ einen \underline{isolierten Punkt (Einsiedler)}. Also existiert $\lim_{x \to 0} sgn(x)$ nicht.
\end{enumerate}\end{myexample}
Besitzt eine Funktion weder isolierte Punkte noch Sprungstellen, so ist sie \underline{stetig}.
\begin{mydef}Eine Funktion $f: D_f \mapsto \mathbb{R}$ heisst \underline{stetig} an der Stelle $a \in D_f$, wenn
\begin{equation}\lim_{x \to a} f(x) = f(a)\end{equation}
\end{mydef}
\begin{myexample}TODO\end{myexample}

\chapter{Differentialrechnung}
\section{Differentialquotient}
Wir betrachten eine stetige Funktion
\begin{equation}f: D_f \mapsto \mathbb{R}\end{equation}
und suchen die Steigung der Tangente an dem Graphen an der Stelle $x_0 \in D_f$.
\\\\TODO\\\\
Die Tangente ist eine spezielle Lage der Sekante. Die Steigung der Sekante ist
\begin{equation}m_s = \frac{\Delta y}{\Delta x} = \frac{f(x_0 + h) - f(x_0)}{h}\end{equation}
, was wir den \underline{Differentialquotienten} nennen. Wird $h$ immer kleiner, so nähert sich die Sekante der Tangente. Also ist die Tangentensteigung
\begin{equation}m_t = \lim_{n \to 0} \frac{f(x_0 + h) - f(x_0)}{h}\end{equation}
\begin{mydef}Wir nennen
\begin{equation}\frac{dy}{dx} := \lim_{n \to 0} \frac{f(x_0 + h) - f(x_0)}{h} \quad \mbox{("dy nach dx")}\end{equation}
den \underline{Differentialquotienten}.\end{mydef}
\begin{myexample}\begin{enumerate}
\item TODO
\item TODO
\item TODO
\end{enumerate}\end{myexample}
Wir kennen den Berührungspunkt $B(x_0 / f(x_0)) = B (x_0 / y_0)$ der Tangente an den Graphen.\\
Ist $y = mx + q$, so wird\\\\
mit B: \begin{eqnarray}y_0 & = & m \cdot x_0 + q \nonumber \\
q & = & y_0 - m \cdot x_0\end{eqnarray}
also
\begin{eqnarray}y & = & mx + y - mx_0 \nonumber \\
y - y_0 & = & mx - mx_0 \nonumber \\
y - y_0 & = & m(x - x_0) \end{eqnarray}
Dies nennen wir die \underline{Punktsteigungsform der Geraden}. Im Beispiel ist $m = 3$ und $B(1/1)$, also
\begin{eqnarray}t:& & y - 1 = 3(x-1) \nonumber \\
t: & & y = 3x -2\end{eqnarray}
\end{document}